\pdfbookmark[0]{Abstract}{Abstract}
\chapter*{Abstract}
\label{chap:abstr}
\vspace{-1cm}

\section*{Abstract (deutsch)}
\label{sec:abstr-de}
\vspace{-1cm}
Ziel der Diplomarbeit ist es, eine funktionstüchtige Satelliten-Ground-Station aufzubauen, um mit 
Satelliten im Amateurfunkband, vor allem auch dem CubeSat des STS1, kommunizieren zu können.\\
Im ersten Schritt muss hierzu die Ground-Station selbst aufgebaut werden. Dazu gehören zum 
Beispiel die Demodulation, Low-Noise-Amplification und ein Software-Defined-Radio. Sobald die 
Ground-Station funktionstüchtig ist, sollen zwei verschiedene Antennen-Typen gebaut und mit der 
Ground-Station betrieben werden, um den besten Antennen-Typ für den Empfang der CubeSat-Daten zu ermitteln. Die empfangenen Daten sollen weiters über eine grafische Benutzeroberfläche 
übersichtlich dargestellt werden können.\\
Im letzten Schritt werden die verschiedenen Antennen charakterisiert und Werte wie die Richtcharakteristik und die Gain ermittelt.

\section*{Abstract (english)}
\label{sec:abstr-en}
\vspace{-1cm}
The goal of the thesis is to establish a functional satellite ground station to communicate with satellites in the amateur radio band, especially the CubeSat STS1.\\
In the first step, the ground station itself needs to be set up. This includes tasks such as demodulation, low-noise amplification, and a software-defined radio. Once the ground station is operational, three different types of antennas will be constructed and operated with the ground station to determine the best antenna type for receiving CubeSat data. The received data should be presented in a clear manner through a graphical user interface.\\
In the final step, the various antennas will be characterized, and values such as directional characteristics and gain will be determined.