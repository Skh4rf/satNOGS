\pdfbookmark[0]{Abstract}{Abstract}
\section*{Abstract}
\label{chap:abstr}
Shortly before his death, Stephen Hawking advised humanity: \glqq Look up at the stars, not down at your feet!\grqq{} This underscores the importance of space exploration. In order to enable investigations in space, specifically in the Low-Earth Orbit, for students at general and vocational higher schools, the TU Wien Space Team launches a satellite in 2024 as a research platform for this purpose into Earth's orbit. To communicate with this satellite, a global network of receiving stations is required, for which a receiving station is developed, constructed, and put into operation as a case study in this thesis.

\subsection*{A Objectives}
The main objective of this thesis is to successfully receive data from existing satellites in the 70cm band. For this purpose, the receiving station, which is put into operation as part of the work, will be integrated into the global satellite ground station network SatNOGS. This will provide satellite operators with extended geographical coverage for communication and improved reliability.

\subsection*{B Implementation}
As main components of the receiving station an antenna array consisting of four directional monofilament helical antennas and a quasi-omnidirectional quadrifilar helical antenna are developed and constructed. For the necessary tracking of the directional antennas, a Yeasu G-5500DC rotor is used, and the required computer control interface is emulated. All remaining components for the receiving station are compactly housed in a switch cabinet.

\subsection*{C Results}
The result of the thesis is a functional receiving station that receives data from scientific satellites in the 70cm band using both developed antenna types. Additionally, the documentation of the time and financial expenditure allows for the evaluation of the profitability of the respective antenna types and the identification of further advantages and disadvantages.