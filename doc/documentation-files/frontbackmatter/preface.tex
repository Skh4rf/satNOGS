\pdfbookmark[0]{Danksagung}{Danksagung}
\chapter*{Danksagung}
\label{chap:Danksagung}
%\vspace*{5cm}
Wir wollen an dieser Stelle allen danken, die uns diese Diplomarbeit ermöglicht haben. Danke an die VFZ dafür, dass die Spirale und Struktur-Elemente so genau gefertigt wurden, und ein großes Dankeschön an die EHG die uns das Metall dafür gesponsert hat. Wir wollen auch SenTech für den Kunststoff danken, den wir als wichtige Strukturelemente (Seitenelemente und Befestigung der Antenne an dem Gerüst) benötigt haben. Wir wollen auch Omicron, der VKW und Liebherr für ihre großzügigen Sponsorings danken. Dieses Projekt wurde auch von Extrudr ermöglicht, welche ein widerstandsfähiges Filament gesponsert haben, womit wir 3D-Drucke anfertigen konnten. Dankend zu erwähnen gilt es auch Lukas Metzler, der neben der großzügigen Bereitstellung des Schaltschranks, des Netzteils, der Schuko-Steckdosen, Maschinen zur Fertigung, verschiedener Fertigungsmaterialien, Kabel und weiterer Einzelteile auch mit seinem technischen Know-How Ideen zur Umsetzung des Projekts gab. Eine weitere inspirierende Person, der wir nicht nur für die temporäre Zurverfügungstellung des HackRF One danken wollen, ist Wilfried Häusle als unser Ansprechpartner bei Omicron und Ortsstellenleiter der Ortsstelle Hofsteig des österreichischen Versuchssenderverband für seine Unterstützung. Nicht zuletzt wollen wir unserem Betreuungslehrer Christian König und allen Lehrpersonen danken die genauso motiviert an diesem Projekt gearbeitet haben wie wir selbst.
Ein großes Dankeschön gilt auch unseren Eltern, welche uns über die ganze Diplomarbeit hinweg immer unterstützt haben wo sie konnten.

Danke dass ihr uns diesen Weg erleichtert habt!
