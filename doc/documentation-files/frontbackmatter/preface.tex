\pdfbookmark[0]{Danksagung}{Danksagung}
\chapter*{Danksagung}
\label{chap:Danksagung}
%\vspace*{5cm}
Wir wollen an dieser Stelle allen danken, die uns diese Diplomarbeit ermöglicht haben.\\ Danke an die VMZ Maschinenbau u. Laserbearbeitung GmbH dafür, dass die Spirale und Strukturelemente so genau gefertigt wurden. Ebenfalls ein großes Dankeschön an die EHG Stahlzentrum GmbH, die uns das Metall dafür gesponsert hat. Wir wollen auch SenTech für den Kunststoff danken, den wir als wichtige Strukturelemente (Seitenelemente und Befestigung der Antenne an dem Gerüst) benötigt haben. Wir wollen zudem der OMICRON electronics GmbH, der illwerke vkw AG und der Liebherr-Werk Nenzing GmbH für ihre großzügigen Sponsorings danken. Dieses Projekt wurde darüber hinaus von Extrudr ermöglicht, welche ein widerstandsfähiges Filament gesponsert haben, womit wir 3D-Drucke anfertigen konnten. Dankend zu erwähnen gilt es zudem Lukas Metzler, der neben der großzügigen Bereitstellung des Schaltschranks, des Netzteils, der Schuko-Steckdosen, von Maschinen zur Fertigung, von verschiedener Fertigungsmaterialien, von Kabel und weiteren Einzelteilen auch mit seinem technischen Know-How Ideen zur Umsetzung des Projekts einbrachte. Eine weitere inspirierende Person, der wir nicht nur für die temporäre Zurverfügungstellung des HackRF One danken wollen, ist Wilfried Häusle als unser Ansprechpartner bei Omicron und Ortsstellenleiter der Ortsstelle Hofsteig des österreichischen Versuchssenderverbandes. Einen Großteil der Konstruktionspläne verdanken wir Dominik Ritter, der diese Arbeit innerhalb kürzester Zeit verrichtete. Nicht zuletzt wollen wir unserem Betreuungslehrer Christian König und allen Lehrpersonen danken, die genauso motiviert an diesem Projekt gearbeitet haben wie wir selbst.
Ein großes Dankeschön gilt auch unseren Eltern, welche uns während unserer Diplomarbeit immer unterstützt haben wo sie nur konnten.

Danke, dass ihr uns diesen Weg erleichtert habt!
