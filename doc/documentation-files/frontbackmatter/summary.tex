\addcontentsline{toc}{section}{Zusammenfassung}
\section*{Zusammenfassung}
Stephen Hawkins riet der Menschheit kurz vor seinem Tod: \glqq Schaut nicht runter zu euren Füßen, sondern rauf zu den Sternen!\grqq{} und unterstreicht damit die Relevanz der Erforschung des Weltalls. Um bereits Schülerinnen und Schülern in allgemein- und berufsbildenden höheren Schulen Untersuchungen im Weltraum, genauer gesagt im Low-Earth-Orbit, zu ermöglichen befördert das TU Wien Space Team im Jahr 2024 einen Satellit als Forschungsplattform für genau diesen Zweck in den Erdorbit. Um mit diesem Satellit kommunizieren zu können, wird ein globales Netzwerk an Empfangsstationen benötigt, für welches als Fallstudie in dieser Diplomarbeit eine Empfangsstation entwickelt, aufgebaut und in Betrieb genommen wird. 

\subsection*{A Aufgabenstellung}
Das Hauptziel dieser Diplomarbeit ist es, erfolgreich Daten von bereits vorhandenen Satelliten im 70cm-Band zu empfangen. Hierfür soll die Empfangsstation, die im Rahmen der Arbeit in Betrieb genommen wird, in das globale Satellitenbodenstationsnetzwerk SatNOGS integriert werden. Dadurch wird den Betreibern von Satelliten eine erweiterte flächenmäßige Abdeckung für die Kommunikation und eine verbesserte Ausfallsicherheit geboten.

\subsection*{B Umsetzung}
Als Hauptbestandteil der Empfangsstation wird sowohl ein Antennenarray aus vier gerichteten monofilaren Helixantennen als auch eine quasi-omnidirektionale quadrifilare Helixantenne entwickelt und aufgebaut. Für die notwendige Nachführung der gerichteten Antennen wird auf einen Yaesu G-5500DC Rotor zurückgegriffen und das dafür benötigte Computer Control Interface emuliert. Die gesamten restlichen Komponenten für die Empfangsstation werden kompakt in einem Schaltschrank untergebracht.  

\subsection*{C Ergebnisse}
Das Ergebnis der Diplomarbeit ist eine funktionsfähige Empfangsstation, die Daten von wissenschaftlichen Satelliten im 70cm-Band mithilfe beider entwickelter Antennentypen empfängt. Darüber hinaus ermöglicht die Dokumentation des zeitlichen und finanziellen Aufwands die Bewertung der Rentabilität der jeweiligen Antennentypen sowie die Identifizierung weiterer Vor- und Nachteile.