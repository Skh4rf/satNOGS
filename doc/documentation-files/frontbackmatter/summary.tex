\addcontentsline{toc}{section}{Zusammenfassung}
\section*{Zusammenfassung}
Es soll eine Bodenstation zum Empfang von Daten wissenschaftlicher Satelliten im 70cm Frequenzband aufgebaut werden. Das Problem ist, dass rund um die Uhr Daten empfangen werden sollen, allerdings befinden sich die Satelliten im LEO, was zur Folge hat, dass sie nicht mit der Erdrotation synchronisiert sind. Dadurch wird ein globales Bodenstations-Netzwerk benötigt, SatNOGS schafft hierbei Abhilfe.
Um die Eignung verschiedener Antennentypen zum Empfang von Daten auszuwerten sollen zwei verschiedene aufgebaut und getestet werden.
Das Ergebnis war eine funktionale Bodenstation sowie eine omnidirektionale Antenne (QHA) und ein direktionales Antennenarray bestehend aus vier Helixantennen.

\subsection*{A Aufgabenstellung}
\begin{itemize}	
	\item Welches Ziel soll erreicht werden?
	
	Das Ergebnis soll eine funktionale Bodenstation sein, welche Daten von wissenschaftlichen Satelliten empfangen kann.
	
	\item Warum und für wen ist das definierte Ziel von Interesse?
	
	Das Ziel ist für das Space Team der Technischen Universität Wien von Interesse, da dadurch von ihrem neuen CubeSat über einen längeren Zeitraum hinweg Daten empfangen werden können.
\end{itemize}

\subsection*{B Umsetzung}
\begin{itemize}
	\item Auf welche fachtheoretischen/-praktischen Grundlagen wurde zurückgegriffen?
	
	Es wurde auf hochfrequenztechnische Grundlagen zum Thema Anpassnetzwerke zurückgegriffen. Weiters musste mit einem Raspberry Pi umgegangen werden, um das SatNOGS-Image darauf zu flashen.
	
	\item 
\end{itemize}

\subsection*{C Ergebnisse}
\begin{itemize}
	\item Welchen Wert hat diese Diplomarbeit für Andere?
	
	Die Diplomarbeit ist im SatNOGS-Netzwerk registriert. Das bedeutet, dass jeder Zugriff auf die Antennen hat und sie nach Wunsch benutzen kann um Daten von Satelliten zu empfangen.
	
	\item 
\end{itemize}