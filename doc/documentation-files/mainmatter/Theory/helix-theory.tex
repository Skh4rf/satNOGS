\section{Monofilare Helixantenne}
Im Folgenden wird die Theorie zur Helixantenne näher beschrieben.

Die Helixantenne ist die in der Satellitenkommunikationstechnik die am Weitesten verbreitete Antennenart. Der Grund hierfür ist die Immunität der zirkularen Polarisation gegenüber Faraday Rotation. Dieses Phänomen findet in der Ionosphäre statt und wurde in (Referenz) bereits näher erklärt.\\
\newline
Die Helixantenne hat weiters verschiedene Operationsmodi. Arbeitet die Antenne im Normal-Mode, so zeigt sie ein omnidirektionales Abstrahlverhalten, wobei Sie senkrecht zur Achse der Antenne gleichmäßig in alle Richtungen strahlt.\\
\newline
Da die Wendelantenne über eine besonders große Bandbreite verfügt, eignet sie sich gut für den Nachbau. Hierbei sind der Durchmesser sowie die Steigung der Spirale von ausschlaggebender Relevanz für die gewünschte Einsatzfrequenz. \cite{Amrhein2019}
\cite{Baird2012VisualEnvironments}


\subsection{Vorteile}
Der große Vorteil von Helixantennen besteht darin, dass diese Antennenbauform die einfachste ist, um eine zirkulare Polarisation zu erzielen. Durch eine zirkulare Polarisation, wie bereits in QUERVERWEIS erwähnt, können sowohl horizontal als auch vertikal polarisierte elektromagnetische Wellen empfangen werden.

\subsection{Funktionsweise und charakteristische Eigenschaften}
Dipol/Ringantenne zusammenhang erklären, zirkulare Polarisation, Zusammenhang mit der Stromverteilung, Erklärung des Abstandes zwischen den Windungen, Durchmesser, Reflektor erklären,

\subsection{Berechnung}
Der Durchmesser der Helix lässt sich mit $D=\frac{\lambda}{\pi}$ berechnen. Dies rührt daher, dass sich mit diesem Durchmesser die Stromverteilung auf der Antenne wie folgt darstellen lässt.