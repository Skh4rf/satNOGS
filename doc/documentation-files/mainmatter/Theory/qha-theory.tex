\section{QFH (Quadrifilare Helixantenne)}
Die QFH ist eine symmetrische Antenne, was bedeutet, dass sie mithilfe eines Baluns auf ein unsymmetrisches Koaxialkabel angepasst werden muss. Mehr dazu in der Subsektion \ref{subsec:baluns}.\\

Die QFH besteht aus einer größeren Schleife, welche unterhalb der gewollten Frequenz resonant ist, und einer kleineren Schleife, welche oberhalb der gewollten Frequenz resonant ist. Die größere Schleife bildet die kapazitive Komponente und die kleinere Schleife die induktive Komponente. Dadurch entsteht eine Phasenverschiebung von (ideal) 90° über die bifilaren Elemente. Werden die Streifen richtig dimensioniert, so ergibt sich eine gute zirkulare Polarisation, weichen sie ab so ergibt sich eine elliptische Polarisation.\\

Das, was die QHA für die Satellitenkommunikation so attraktiv macht, ist zum einen ihre zirkulare Polarisation, und zum anderen ihre kompakte Bauweise. Diese macht sie einfach transportierbar. Zudem hat sie omnidirektionale Abstrahlcharakteristiken, was den Einsatz eines Rotors nichtig macht. Da die QHA entlang des Horizonts ihren größten Antennengewinn aufweist, macht sie das zu einer guten Wahl für die Weltraumkommunikation\cite{qfh_w3kh_nodate}.
