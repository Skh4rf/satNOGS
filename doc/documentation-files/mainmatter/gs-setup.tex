\chapter{Einrichten des SatNOGS Client}
\label{chap:gs-setup}
Der SatNOGS Client bildet das Herzstück der gesamten Empfangsstation, ist für die Ausführung und Koordinierung der, mit der Empfangstation geplanten, Observationen zuständig und bildet die Schnittstelle zur SatNOGS Datenbank. Das SatNOGS-Client Script kann auf einer beliebigen Linux Distribution installiert und ausgeführt werden. Für die, wie in diesem Fall, Anwendung eines Raspberry Pi 4  Einplatinencomputers zur Ausführung des Client Scripts empfiehlt sich, dass speziell für den SatNOGS Client entwickelte Raspbian SatNOGS Image, zu verwenden. Dieses angepasste Image für das Raspbian Betriebssystem sorgt dafür, dass alle benötigten Softwarepakete und das SatNOGS setup script vorinstalliert vorinstalliert sind und der SSH-Server auf dem Raspberry Pi 4 aktiviert wird. Auf die Installation wird nicht näher eingegangen sondern auf die Anleitung im SatNOGS Wiki \cite{noauthor_raspberry_nodate} verwiesen. 

\section{grundlegende Konfiguration}
\subsection{Verbinden zum Raspberry PI 4}
Um Einstellungen am Raspberry Pi 4 vornehmen zu können wird auf diesen über SSH zugegriffen. Dazu muss das Gerät welches auf den Raspberry Pi 4 zugreift am besten mit dem selben Netzwerk verbunden sein wie der Raspberry selbst. Die einfachste Vorgehensweise um sich mit dieser Voraussetzung zum Raspberry Pi 4 über SSH zu verbinden, besteht mit einem Windows System darin, in der Eingabeaufforderung den Befehl \textbf{ssh *hostname*} auszuführen. Der Hostname wird im Zuge der Installation des Betriebssystems vergeben und lautet in diesem Fall \textbf{qfh}. Wurde ein Passwort im Zuge der Installation eingegeben, so erscheint anschießend eine Aufforderung dieses einzugeben. Nach der Eingabe wird bei erfolgreicher Verbindung folgende Information zurückgegeben:

\begin{figure} [H]
	\centering
	\includegraphics[width=\linewidth]{../ref/successfull_login.png}
	\caption{Erfolgreiche Verbindung über SSH}
	\label{fig:htrl-uhf(test)successfulllogin}
\end{figure}

\subsection{Hauptmenü}
Mithilfe des Befehls \textbf{sudo satnogs-setup} öffnet sich das Hauptmenü des SatNOGS Client über welches alle Einstellungen vorgenommen werden können.

\begin{figure} [H]
	\centering
	\includegraphics[width=.5\linewidth]{../ref/mainmenu.png}
	\caption{SatNOGS Client: Hauptmenü}
	\label{fig:mainmenu}
\end{figure}

Wurden alle Einstellungen vorgenommen so können diese mit \textbf{Apply} angewendet und gespeichert und angewendet werden.

\subsection{Einstellungen}
Wird im Hauptmenü der Reiter \textbf{Basic} mit \textbf{Enter} ausgewählt, so erscheint folgendes Fenster über welches alle grundlegenden Einstellungen zum Betrieb einer Empfangsstation vorgenommen werden können.

\begin{figure} [H]
	\centering
	\includegraphics[width=.75\linewidth]{../ref/basic_configurations.png}
	\caption{SatNOGS Client: grundlegende Einstellungen}
	\label{fig:basic_configurations}
\end{figure}

\subsubsection{SATNOGS\textunderscore API\textunderscore TOKEN}
Der API-Schlüssel, welcher mit dem SatNOGS-Benutzerkonto verknüpft ist, wird von der Empfangsstation benötigt sich gegenüber der SatNOGS-Datenbank zu identifizieren und authentifizieren. Er kann auf dem Benutzerprofil der SatNOGS-Netzwerk Webseite durch das Klicken auf den in der folgenden Grafik mit 1 markierten Button an der mit 2 markierten Stelle abgelesen werden.

\begin{figure} [H]
	\centering
	\includegraphics[width=.75\linewidth]{../ref/apitoken.png}
	\caption{SatNOGS-Netwerk Homepage: API-Schüssel ermitteln}
	\label{fig:apikey}
\end{figure}

\subsubsection{SATNOGS\textunderscore SOAPY\textunderscore RX\textunderscore DEVICE}
Mithilfe dieses Einstellungsparameters wird dem SatNOGS Client das verwendete SDR mitgeteilt. In diesem Fall entspricht das SDR dem, im späteren Kapitel \ref{sec:sdr} genauer Erläuterten, RTL-SDR Blog v3 und die dementsprechende einzugebende Treiber-Kennzeichnung lautet \textbf{driver=rtlsdr} \cite{noauthor_satnogsclient_nodate}. Weitere Treiber-Kennzeichnungen können auf der Software Defined Radio Seite des SatNOGS Wiki \cite{noauthor_software_nodate} nachgeschlagen werden.

\subsubsection{SATNOGS\textunderscore ANTENNA}
Über diesen Einstellungsparameter wird entsprechend dem verwendeten SDR und dem passenden Parameter der Modus des SDRs konfiguriert. Für das RTL-SDR Blog v3 entspricht das dem Parameter \textbf{RX} \cite{noauthor_satnogsclient_nodate}. Entsprechende Parameter für weitere SDR Typen können auf der Software Defined Radio Seite des SatNOGS Wiki \cite{noauthor_software_nodate} nachgeschlagen werden. 

\subsubsection{SATNOGS\textunderscore RX\textunderscore SAMP\textunderscore RATE}
Dieser Parameter legt die Abtastrate des SDRs fest. Für das RTL-SDR Blog v3 wird eine Abtastrate von 2.048 Megasamples pro Sekunde empfohlen \cite{noauthor_satnogsclient_nodate}, was über den Parameter 2.048e6 eingestellt werden kann. Auch hier finden sich für andere SDRs Informationen auf der Software Defined Radio Seite des SatNOGS Wiki \cite{noauthor_software_nodate}. 

\subsubsection{SATNOGS\textunderscore STATION\textunderscore ELEV}
Für diesen Parameter ist es notwendig die Höhenlage der Empfangsstation zu ermitteln. Es empfiehlt sich dazu eine interaktive topografische Karte, wie de-at.topographic-map.com, zu verwenden. Für den Standort der HTL Rankweil entspricht dieser Wert, was zugleich auch dem Parameter entspricht, 459 Meter. Dieser Wert wird für die Berechnungen der passierenden Satelliten und dessen Elevation- und Azimutalwinkel verwendet.

\subsubsection{SATNOGS\textunderscore STATION\textunderscore ID}
Die Station-ID wird zur eindeutigen Identifikation der Empfangsstation im SatNOGS-Netzwerk verwendet und kann auf der Seite des Benutzerprofils auf der Homepage des SatNOGS-Netzwerks abgelesen werden. 

\begin{figure} [H]
	\centering
	\includegraphics[width=.75\linewidth]{../ref/stationid.png}
	\caption{SatNOGS-Netzwerk Homepage: Station-ID ermitteln}
	\label{fig:stationid}
\end{figure}

\subsubsection{SATNOGS\textunderscore STATION\textunderscore LAT}
Dieser Parameter gibt den Breitengrad der Position der Empfangsstation an. Dieser Wert wird für die Berechnungen der passierenden Satelliten und dessen Elevation- und Azimutalwinkel verwendet.

\subsubsection{SATNOGS\textunderscore STATION\textunderscore LON}
Dieser Parameter gibt den Längengrad der Position der Empfangstation an. Dieser Wert wird für die Berechnungen der passierenden Satelliten und dessen Elevation- und Azimutalwinkel verwendet.

\section{erweiterte Konfiguration} 
Wird im Hauptmenü der Reiter \textbf{Advanced} mit \textbf{Enter} ausgewählt, so erscheint folgendes Fenster über welches erweiterte Einstellungen vorgenommen werden können. 

\begin{figure} [H]
	\centering
	\includegraphics[width=.5\linewidth]{../ref/advanced_settings.png}
	\caption{SatNOGS Client: erweiterte Einstellungen}
	\label{fig:advanced_settings}
\end{figure}

Die wichtigsten Einstellungen die hier vorgenommen werden können sind das definieren der Verstärkung des SDRs, das Einrichten einer Unterstützung für die Verwendung eines Rotors und das automatische Aktivieren von Bias-T während einer Observation.

\subsection{Verstärkung des SDRs einstellen}
Um die Verstärkung des SDRs einzustellen müssen zuerst die möglichen Parameter ermittelt werden, was durch das ausführen des Befehls \textbf{rtl\textunderscore test} in der SSH gemacht werden kann. 

\begin{figure} [H]
	\centering
	\includegraphics[width=\linewidth]{../ref/supportedgain.png}
	\caption{SatNOGS Client: unterstützte Gain Werte ermitteln}
	\label{fig:supportedgain}
\end{figure}

Wurde aus der Ausgabe, wie in obiger Darstellung gezeigt, der gewünschte Wert der Verstärkung gewählt, so kann dieser in den erweiterten Einstellungen unter dem Reiter \textbf{Radio} mithilfe des Parameters \textbf{SATNOGS\textunderscore RF\textunderscore GAIN} eingestellt werden. 

\subsection{Rotor-Unterstützung implementieren}
Für die Implementierung einer Rotorsteuerung müssen in den erweiterten Einstellungen unter dem Reiter \textbf{Rotator} drei Parameter angepasst werden. 

Der erste Parameter \textbf{SATNOGS\textunderscore ROT\textunderscore ENABLED} aktiviert die Verwendung der konfigurierten Rotorsteuerung. 

Mithilfe des Parameters \textbf{SATNOGS\textunderscore ROT\textunderscore MODEL} wird das Model der Rotorsteuerung festgelegt. Im Fall der im späteren Kapitel \ref{sec:gs232emulation} erläuterten GS232A/B Steuerung wird für den Parameter \textbf{ROT\textunderscore MODEL\textunderscore GS232B} verwendet \cite{noauthor_satnogsclient_nodate}. 

\textbf{SATNOGS\textunderscore ROT\textunderscore BAUD} legt die Baud-Rate der Kommunikation mit der Steuerung fest und beträgt im Fall der im späteren Kapitel \ref{sec:gs232emulation} erläuterten Steuerung 9600.

\subsection{Aktivieren von Bias-T}
Das Aktivieren von Bias-T ist abhängig vom verwendeten SDR und wird nun nur für das RTL-SDR Blog v3 erläutert. Um die softwaremäßigen Voraussetzungen für die Verwendung von Bias-T am RTL-SDR Blog v3 zu erfüllen müssen im ersten Schritt folgende Befehle in der SSH ausgeführt werden:
\newline 1. \textbf{sudo apt install cmake}
\newline 2. \textbf{git clone https://github.com/rtlsdrblog/rtl\textunderscore biast}
\newline 3. \textbf{cd rtl\textunderscore biast}
\newline 4. \textbf{mkdir build}
\newline 5. \textbf{cd build}
\newline 6. \textbf{cmake ..}
\newline 7. \textbf{make}

Diese Schritte ermöglichen die Kontrolle der Bias-T Funktion des RTL-SDR Blog v3 und müssen nur einmal durchgeführt werden. Sie ermöglichen die Kontrolle der Funktion über folgende Befehle:
\newline Einschalten: \textbf{/rtl\textunderscore biast/build/src/rtl\textunderscore biast -b 1}
\newline Ausschalten: \textbf{/rtl\textunderscore biast/build/src/rtl\textunderscore biast -b 0}

Diese Befehle können nun in den erweiterten Einstellungen unter dem Reiter Scripts als Pre- und Post-Observation Script festgelegt werden. Dazu wird der Einschalt- sowie Ausschaltbefehl mit \textbf{/home} am Anfang, aufgrund des Ursprungsverzeichnisses des SatNOGS Client, erweitert werden.