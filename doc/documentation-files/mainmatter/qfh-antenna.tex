\chapter{Quadrifilare Helixantenne (QHA)}
Die Quadrifilare Helixantenne ist eine Abwandlung der monofilaren Helixantenne und unterscheidet sich von dieser in mehreren wichtigen Punkten.

Zum einen ist sie um einiges kleiner als die monofilare Helix. Zum anderen wird sie als omnidirektionale Antenne für Satellitenkommunikation verwendet, da sie, wie die monofilare Helix, zirkular polarisiert ist. Allerdings zeigt sie entlang des Horizonts sowie direkt nach oben einen besseren Antennengewinn als in andere Richtungen.

\section{Funktionsweise}
Die QHA ist eine symmetrische Antenne, was bedeutet, dass sie mithilfe eines Baluns auf ein unsymmetrisches Koaxialkabel angepasst werden muss. Mehr dazu in Kapitel QUERVERWEIS.

Die QFH besteht aus einer größeren Schleife, welche unterhalb der gewollten Frequenz resonant ist, und einer kleineren Schleife, welche oberhalb der gewollten Frequenz resonant ist. Die größere Schleife bildet die kapazitive Komponente und die kleinere Schleife die induktive Komponente. Dadurch entsteht eine Phasenverschiebung von (ideal) 90° über die bifilaren Elemente. Werden die Streifen richtig dimensioniert, so ergibt sich eine gute zirkulare Polarisation, weichen sie ab so ergibt sich eine elliptische Polarisation.

\section{Design}
stl files, beschreibung der form

BILD DER QFH

\section{Realisierung}
Metallstreifen, metallrundlinge

BILD DER QFH

\section{Messungen}