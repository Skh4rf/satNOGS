\chapter{Yaesu-G5500 Rotor}
Rotoren werden in der Antennentechnik eingesetzt, um die Kommunikation mit nicht-geostationären Satelliten zu ermöglichen, wobei Antennentypen mit hoher Richtwirkung verwendet werden (z.B. Helix-Antenne). In der Regel wird durch diese Methode eine höhere Verstärkung erzielt, als es mit Antennen möglich ist, die den gesamten Horizont abdecken und daher keine Nachführung erfordern. 
\section{Azimut und Elevation}
Die Ausrichtung des Rotors und somit auch die der Antenne wird mithilfe der Angabe von zwei Winkel - Azimut und Elevation - im Gradmaß des astronomischen Horizont-Koordinatensystem angegeben, wobei das Azimut der auszurichtende Himmelsrichtung und die Elevation der vertikalen Ausrichtung entspricht. Beide Winkelscheitel sind dabei durch Position des Rotors definiert. Als Referenzschenkel für das Azimut dient die Linie vom Winkelscheitel nach Norden. Der Referenzschenkel für die Elevation bildet der Zenit am Ort des Rotors. Die Elevation entspricht somit dem Komplementärwinkel vom Referenzschenkel zur Linie welche den Winkelscheitel mit dem auszurichtenden Punkt verbindet.

\begin{figure}[H]
	\centering
	\includegraphics[width=6cm]{../ref/Azimuth-Altitude_schematic_satellit.png}
	\label{fig:Wndg_aufgerollt}
\end{figure}

\url{https://de.wikipedia.org/wiki/Azimut#/media/Datei:Azimuth-Altitude_schematic.svg}