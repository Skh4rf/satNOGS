\chapter{Conclusio}

Mit dem Abschluss der Diplomarbeit wurden die gesteckten Ziele erfolgreich erreicht und wichtige Erkenntnisse gewonnen. Die Arbeit präsentiert ein funktionierendes Konzept für den Bau einer Empfangsstation im globalen Satellitenkommunikationsnetzwerk SatNOGS, das die flächenmäßige Abdeckung erweitert und die Ausfallsicherheit erhöht.

Die Erfahrungen aus dem Bau der quadrifilaren Helixantenne erwiesen sich als äußerst wertvoll beim Bau der monofilaren Helixantenne, was zu einem ausgereiften Antennenarray führte. Unter Berücksichtigung des finanziellen Aspekts ist festzustellen, dass selbst bei geringerem finanziellen Aufwand, wie es bei einer quadrifilaren Helixantenne der Fall ist, eine zuverlässige Antenne und somit eine erschwingliche Empfangsstation für den Empfang von Daten wissenschaftlicher Satelliten realisiert werden kann.

In Anbetracht der Kosteneffizienz der quadrifilaren Helixantenne ist es sinnvoll, ihre Entwicklung fortzusetzen. Dies könnte durch eine eingehende Untersuchung und Analyse der Abweichungen zwischen der gemessenen Richtwirkung und den Simulationsergebnissen erfolgen, sowie durch die Prüfung der Auswirkungen der Form des Materials zur Modellierung der Schleifen auf die Antenneneigenschaften.

Die im Pflichtenheft festgelegte Dekodierung und Darstellung der empfangenen Daten für den CubeSat des TU Wien Space Team konnte nicht durchgeführt werden, da das dafür benötigte Protokoll noch nicht vollständig vom Space Team entwickelt wurde.