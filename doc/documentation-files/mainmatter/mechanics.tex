\chapter{Empfangsstation}
Als Empfangsstation wird das Gesamtsystem bezeichnet welches alle Komponenten beinhaltet die für den Empfang von Satellitendaten benötigt werden. In diesem Kapitel soll das Zusammenspiel der Komponenten und die Funktionsweise bisher noch unbehandelter Komponenten geklärt werden.

\section{Betriebsmittelkennzeichnung}
\label{sec:bmk}
Um alle Objekte innerhalb des Systems jederzeit eindeutig identifizieren zu können werden diese nach OVE EN IEC 81346-2 einer Referenzkennzeichnung unterzogen. Die benötigten Kennbuchstaben sind:

\begin{tabular}{|c|l|}
	\hline
	\textbf{Kennbuchstabe} & \textbf{Zweck oder Aufgabe nach EN IEC 81346-2} \\
	\hline
	A & zwei oder mehr Zwecke - kein identifizierbarer Hauptzweck \\
	\hline
	K & Verarbeiten von Signalen oder Informationen \\
	\hline
	M & Bereitstellen von mechanischer Energie zu Antriebszwecken \\
	\hline
	T & Umwandeln von Energie unter Beibehaltung der Energieart \\
	\hline
	W & Leiten oder Führen von Energie \\
	\hline
	X & Verbinden von Objekten \\
	\hline
\end{tabular}

Zur weiteren Differenzierung werden Ortskennzeichen angewandt. Die verwendeten Symbole und ihre Bedeutung lauten dadurch:

\begin{tabular}{|c|l|}
	\hline
	\textbf{Symbol} & \textbf{Bedeutung} \\
	\hline
	- & Betriebsmittel \\
	\hline
	+ & Ort \\
	\hline
\end{tabular}

Im Zuge dieses Kennzeichnungsprozesses ergibt sich folgende Zuordnung:

\begin{tabular}{|l|c||l|c|}
	\hline
	\textbf{Betriebsmittel} & \textbf{Kennzeichnung} & \textbf{Betriebsmittel} & \textbf{Kennzeichnung} \\
	\hline
	Schaltschrank & -A01 & Schuko 2 & +A01-X02 \\
	\hline
	Helix-Gerüst (\ref{subsec:helix_geruest}) & -A03 & Rotor (Azimut) (\ref{sec:yaesug5500dc}) & +A03-M01 \\
	\hline
	QFH-Antenne (\ref{chap:qfh}) & -T01 & Rotor (Elevation) (\ref{sec:yaesug5500dc}) & +A03-M02 \\
	\hline
	Raspberry Pi 4 QUERVERWEIS & +A01-A02 & RF-Power-Combiner (\ref{chap:helix}) & +A03-T04 \\
	\hline
	Rotor-Controller (\ref{sec:yaesug5500dc}) & +A01-K01 & Helix-Antenne 1 (\ref{chap:helix}) & +A03-T11 \\
	\hline
	GS232-Interface Emulation (\ref{sec:gs232emulation}) & +A01-K02 & Helix-Antenne 2 (\ref{chap:helix}) & +A03-T12 \\
	\hline
	SDR QUERVERWEIS & +A01-K03 & Helix-Antenne 3 (\ref{chap:helix}) & +A03-T13 \\
	\hline
	Netzteil & +A01-T03 & Helix-Antenne 4 (\ref{chap:helix}) & +A03-T14 \\
	\hline
	Netzkabel & +A01-W01 & Antennenkabel Helix 1 & +A03-W08 \\
	\hline
	USB-Kabel & +A01-W02 & Antennenkabel Helix 2 & +A03-W09 \\
	\hline
	5V-Kabel & +A01-W03 & Antennenkabel Helix 3 & +A03-W10 \\
	\hline
	DIN-Kabel & +A01-W04 & Antennenkabel Helix 4 & +A03-W11 \\
	\hline
	Azimut-Kabel & +A01-W05 & Antennenkabel Array & +A03-W12 \\
	\hline
	Elevation-Kabel & +A01-W06 & LNA  & +T01-T02 \\
	\hline
	Schuko1 & +A01-X01 & Antennen-Kabel & +T01-W07 \\
	\hline
\end{tabular}

Die in vorhergehenden Kapiteln noch nicht erwähnten Betriebsmittel und ihre Funktion werden nun im Zuge dieses Kapitels erläutert. 

\section{Schaltschrank (+A01)}
Die Aufgabe des Schaltschranks ist es, abgesehen von der Antennen, dem Rotor und des LNAs alle weiteren Komponenten der Empfangsstation vor UV-Strahlung und Wetter zu schützen. Ein weitere positiver Effekt der sich dadurch ergibt, ist die Flexibilität die ein solcher Schaltschrank bietet. Positionswechsel der Bodenstation erfordern somit nur den Transport der Antenne und des Schaltschranks und können mit geringerer Anzahl an ein- und ausstecken der Kabel durchgeführt werden.

Als Schaltschrank wird ein Stahlblech-Wandgehäuse der Argentina Reihe des Unternehmens IDE ELECTRIC, S.L. mit Außenmaßen von 400mm Höhe, 400mm Breite und 250mm Tiefe verwendet. Mit einer im Datenblatt \cite{ide_electric_sl_datenblatt_nodate} angegebenen IP66 Schutzklasse und UV-Schutz-Beschichtung erfüllt er die Voraussetzungen, um den von der Umwelt ausgehenden mechanischen Belastungen standzuhalten. Die erste Kennziffer der IP66 Schutzklasse garantiert Staubdichtigkeit und die zweite Kennziffer Schutz gegen starkes Strahlwasser und Überflutung \cite{lienig_elektronische_2014} (Seite 53).

\subsection{Befestigung der Komponenten}
Zur Befestigung der Komponenten dient eine mit dem Schaltschrank mitgelieferte 2 Millimeter Montageplatte aus verzinktem Stahlblech. Für die Montage der Schuko-Steckdosen, des Netzteils, der GS232-Emulation und des Raspberry Pi 4 wird auf dieses Stahlblech eine DIN-Schiene geschraubt, welche das einfache Aufstecken dieser Komponenten ermöglicht. Für den Arduino Uno der GS232-Emulation und den Raspberry Pi 4 wurden dafür passende Gehäuse erworben. Die Schuko-Steckdosen und das Netzteil sind bereits für diese Montageart kompatibel. 

Der Rotor-Controller ist die einzige Komponente im Schaltschrank welche nicht über eine DIN-Schiene befestigt wird. Für die Befestigung des Rotors wurde mithilfe eines 2 Millimeter dicken Eisenblechs ein Regal gebogen auf welchem der Rotor-Controller Platz findet. Um den Controller gegen verrutschen zu sichern wird er mithilfe eines gebogenen viereckigen Eisenringes festgespannt. Der Vorteil dieser Befestigung ist, dass das Gehäuse des Rotor-Controllers nicht modifiziert werden musste, der Rotor-Controller mit wenig Aufwand aus dem Schaltschrank entfernt werden kann und dennoch fixiert an seinem Platz ist. Um die Sicherheit der Anwender und Anwenderinnen des Schaltschrankes zu gewährleisten, ist dieser zusätzlich an zwei Erdungspunkten mit dem Schutzleiter des Netzkabels verbunden.

\begin{figure}[H]
	\centering
	\includegraphics[width=0.7\linewidth]{../ref/Schaltschrank_Befestigung.jpg}
	\caption{Befestigungskonzept im Schaltschrank}
	\label{fig:schaltschrankbefestigung}
\end{figure}

\subsection{Anschlüsse am Schaltschrank}
Der Schaltschrank verfügt an der Außenseite über keinerlei Anschlüsse. Die Kabel für die Stromversorgung, Antenne und Rotoren verlaufen über Kabelverschraubungen nach außen um dort direkt verwendet werden zu können. 

\begin{figure}[H]
	\centering
	\includegraphics[width=0.7\linewidth]{../ref/Schaltschrank_Anschluss.jpeg}
	\caption{Anschlüsse am Schaltschrank}
	\label{fig:schaltschrankanschluesse}
\end{figure}

Die unbeschriftete Schraubverbindung in der Abbildung dient als Reserve für mögliche Erweiterungen.

\subsection{Standfüße}
Um den Schaltschrank flexibel platzieren zu können wurden zwei Standfüße aus 10 Millimeter Chromstahlblech zusammengeschweißt und mit Gummifüßen versehen. 

\begin{figure}[H]
	\centering
	\includegraphics[width=9cm]{../ref/Schaltschrank_Fuss.jpeg}
	\caption{Standfuß für den Schaltschrank}
	\label{fig:schaltschrankfuss}
\end{figure}

Das Ergebnis ist ein modular platzierbarer und vielseitiger Schaltschrank:

\begin{figure}[H]
	\begin{minipage}[b]{.4\linewidth}
		\includegraphics[width=\linewidth]{../ref/Schaltschrank_stehend_vorne.jpeg}
		\label{fig:schaltschrankstehendvorne}
		\caption{Schaltschrank stehend (Ansicht vorne)}
	\end{minipage}
	\hspace{.1\linewidth}% Abstand zwischen Bilder
	\begin{minipage}[b]{.4\linewidth} % [b] => Ausrichtung an \caption
		\includegraphics[width=\linewidth]{../ref/Schaltschrank_stehend_hinten.jpeg}
		\label{fig:schaltschrankstehendhinten}
		\caption{Schaltschrank stehend (Ansicht hinten)}
	\end{minipage}
\end{figure}

\section{Netzteil (+A01-T03)}
Für die Stromversorgung des Raspberry Pi 4 sowie des im GS232A/B verbauten Arduino Uno wird das Schaltnetzteil S8VK-G01505 der OMRON Corporation verwendet. Das Netzteil kann mit einer Eingangsspannung von 100 bis 240 Volt und 50 Hertz betrieben werden und sorgt für eine Ausgangsspannung von 5 Volt. Mit einer maximalen Ausgangsleistung von  15 Watt ist es ausreichend groß dimensioniert um sowohl den Raspberry Pi 4 \cite{noauthor_power_nodate} als auch den Arduino Uno \cite{noauthor_r3_nodate} mit Strom zu versorgen. \cite{noauthor_s8vk-g01505_nodate}

\section{Netzkabel (+A01-W01)}
Das Netzkabel, mit einer Länge von 4.6 Metern außerhalb des Schaltschranks, versorgt die gesamte Empfangsstation mit Energie. Voraussetzung zum Betrieb der Empfangsstation ist es also, das Netzkabel mit einer Spannungsquelle (100 bis 120 oder 200 bis 240 Volt und 50 Hz) zu verbinden. Das verwendete Kabel ist ein H07BQ-F 3G1.5 der PATELEC Group \cite{noauthor_cables_nodate} und weißt dankt dem verwendeten Mantelmaterial Polyurethan eine gute UV-Beständigkeit auf \cite{noauthor_polyurethan_nodate}.

\section{USB-Kabel (+A01-W02)}
Das USB-Kabel ist ein USB-A zu USB-B Kabel und verbindet den Raspberry Pi 4 (USB-A) mit dem Arduino Uno (USB-B) der GS232-Interface Emulation. Abgesehen von der Kommunikation zwischen dem Raspberry Pi 4 und dem Arduino Uno, erfolgt auch die Stromversorgung des Arduinos über das USB-Kabel. Es handelt sich bei dem USB-Kabel um ein handelsübliches Kabel aus dem Eigenbestand ohne spezielle Eigenschaften, da es im Schaltschrank keinerlei mechanischen Belastung ausgesetzt ist.

\section{5V-Kabel (+A01-W03)}
Die Aufgabe des 5V-Kabels ist die Energieübertragung vom Netzteil zum Raspberry Pi 4. Die verwendete rot, schwarze Zwillingsleitung wird dazu vom Ausgang des Netzteils zu den auf dem Gehäuse des Netzteils angebrachten Verbindungsklemmen geführt. Der Zwischenschritt über die Verbindungsklemmen ermöglicht somit nicht nur eine leichteres Ein- und Ausbauen der Komponenten, sondern fördert auch die Skalierbarkeit im Bezug auf den Einbau weiterer elektronischer Komponenten welche eine 5 Volt Versorgungsspannung benötigen.

\begin{figure}[H]
	\centering
	\includegraphics[width=.25\linewidth]{../ref/Anschlussklemmen5V.jpeg}
	\caption{Anschlussklemmen 5 Volt Versorgung}
	\label{fig:anschlussklemmen}
\end{figure}

\section{DIN-Kabel (+A01-W04)}
Als DIN-Kabel wird in diesem Fall jenes Kabel bezeichnet, welches zur Verbindung der GS232-Interface Emulation zum G-5500DC Controller verwendet wird und an einem Ende einen 8-Pol DIN-Rundstecker aufweist. Die Zuordnung der Farben des Kabels zu den PIN-Nummern des Steckers, welche im Kapitel \ref{subsubsec:emulation_analoge_schnittstelle} genauer erläutert wurde, sowie die Zuordnung zu den Bezeichnungen auf dem Arduino-Shield, welche im Kapitel \ref{subsec:hardware} erklärt wurde, lautet wie folgt:

\begin{tabular}{c|c|c}
	\textbf{Farbe} & \textbf{Pin-Nummer} & \textbf{Bezeichnung am Arduino-Shield} \\
	\hline
	Grün & 1 & V \\
	Gelb & 2 & R \\
	Grau & 3 & U \\
	Weiß & 4 & L \\
	Blau & 5 & D \\
	Pink & 6 & H \\
	Rot & 7 & Vin \\
	Braun & 8 & GND \\
\end{tabular}

\section{Azimut-Kabel (+A01-W05)}
\label{sec:azimutkabel}
Das Azimut-Kabel dient zur Steuerung und Messung des Azimut-Rotors und wird im Kapitel \ref{subsubsec:kontrollkabel} als Kontrollkabel bezeichnet und seine Funktion genauer beschrieben. Mit einem Leiterquerschnitt von je 1.5 Millimetern erfüllt es die in Kapitel \ref{subsubsec:kontrollkabel} genannten Vorgaben und ist aufgrund des UV-Stabilisierten Außenmantels aus Polyvinylchlorid auch gegen irreparable Schäden durch Sonneneinstrahlung geschützt. Die Länge des Kabels außerhalb des Schaltschranks beträgt 4.8 Meter. 

\section{Elevation-Kabel (+A01-W06)}
Das Elevation-Kabel ist das Pendant zum Azimut-Kabel und dient der Steuerung und Messung des Elevation-Rotors. Die Eigenschaften entsprechen dem Kapitel \ref{sec:azimutkabel}.

\section{Schuko 1 (+A01-X01)}
\label{sec:schuko1}
Die Schutzkontaktsteckdose dient der Energieversorgung von Geräten mit Netzstecker. Das verwendete Modell SD35DEA des Unternehmen SHC GmbH verfügt über eine grüne Status-LED, die leuchtet, sobald das Netzkabel des Schaltschranks an eine Energiequelle angeschlossen wird. Die Schutzkontaktsteckdose Schuko 1 ist als Reserve für mögliche weitere Geräte oder für temporäre Anwendungen wie z.B. das Laden eines Notebooks oder dem Ausleuchten des Schaltschranks eingebaut.

\section{Schuko 2 (+A01-X02)}
Die Schutzkontaktsteckdose Schuko 2 weist dieselben technischen Eigenschaften wie die Schutzkontaktsteckdose 1, welche in Kapitel \ref{sec:schuko1} beschrieben wird, auf. Die Schutzkontaktsteckdose Schuko 2 ist für die Energieversorgung des Yeasu G-5500DC Controllers vorgesehen, kann allerdings auch für jegliche andere Geräte verwendet werden.

\section{Antennenkabel Helix und Array (+A03-W09 bis W12)}
Die Antennenkabel +A03-W09 bis W12 des Typ RG58 C/U führen von den Antennen zum RF-Combiner mit einer Länge von insgesamt 1.25m pro Kabel. Entsprechend der RG58 C/U Bezeichnung weisen die Kabel einen Wellenwiderstand von 50 Ohm sowie \cite{noauthor_rg_nodate} eine Dämpfung von 0.3 Dezibel pro Meter bei 400 Megahertz \cite{noauthor_vergleich_nodate}. 

\section{LNA (+T01-T02)}
Die Leistung einer Antenne lässt sich in einer Gütezahl angeben die als gain-to-noise-temperature, kurz G/T, bezeichnet wird. Diese G/T ist eine positive Zahl die mit steigender Performanz der Antenne zunimmt. Um diese zu verbessern muss entweder der Gewinn der Antenne vergrößert oder der relative Rauschanteil im Signal verringert werden. Weil das Vergrößern des Gewinns einer omnidirektionalen Antenne sich als sehr schwierig gestaltet und in den meisten Fällen in einer gerichteten Antenne mündet, muss der Rauschanteil abnehmen. Um dies zu tun wird das Wideband LNA von RTL-SDR.com in der Signalkette direkt nach der Antenne verwendet. Im Vergleich zum SDR, welches ursprünglich alleine die Aufgabe hatte das Signal der Antenne zu Verstärken, weist das Wideband LNA mit 0.52 Dezibel bei 800 MHz eine sehr geringe Rauschzahl auf. \cite{noauthor_new_nodate} \cite{noauthor_omnidirectional_nodate}

\begin{figure}[H]
	\centering
	\includegraphics[width=.75\linewidth]{../ref/wideband_lna_gain.png}
	\caption{Verstärkung des Wideband LNAs in Abhängigkeit der Frequenz \cite{noauthor_new_nodate}}
	\label{fig:wideband_lna_gain}
\end{figure}

Die Verstärkung des Wideband LNAs entspricht laut gezeigter Grafik des Herstellers etwa 22.5 Dezibel bei 433 Megahertz.

\section{Antennen-Kabel (+T01-W07)}
Das Antennen-Kabel +T01-W07 wird für die Verbindung des SDR mit dem Ausgang des LNAs verwendet und weist eine Länge von insgesamt 3 Metern auf. Das Kabel entspricht dem Typ RG174 und weist somit neben einer Wellenimpedanz von 50 Ohm eine Streckendämpfung von 0.7 Dezibel pro Meter bei 432 Megahertz auf \cite{noauthor_dunnes_nodate}.

\section{Prinzipschaltplan}
Der nachfolgende Prinzipschaltplan dient dem Zweck, Anwenderinnen und Anwendern einen guten Überblick über die verbaute Hardware und dessen Verkabelung zu bekommen. Hierzu wurden die, im Rahmen der Betriebsmittelkennzeichnung in Kapitel \ref{sec:bmk}, festgelegten Kennzeichnungen im Prinzipschaltplan berücksichtigt, sowie die Komponenten in Realität dementsprechend beschriftet.

\begin{landscape}
	\begin{figure}
		\centering
		\includegraphics[width=\linewidth]{../ref/Prinzipschaltplan.jpg}
		\caption{Prinzipschaltplan der Empfangsstation}
		\label{fig:prinzipschaltplan}
	\end{figure}
\end{landscape}

\section{Observation planen}
Um eine Observation mit der Empfangsstation planen zu können, wird Benutzerkonto im SatNOGS Netzwerk mit der entsprechenden Berechtigung benötigt. Folgende Matrix erläutert die möglichen Aktionen die gemäß der Art des Benutzerkontos durchgeführt werden können.

\begin{figure} [H]
	\centering
	\includegraphics[width=\linewidth]{../ref/network_permission_matrix.png}
	\caption{SatNOGS Netzwerkberechtigungsmatrix}
	\label{fig:networkpermissionmatrix}
\end{figure}


	




