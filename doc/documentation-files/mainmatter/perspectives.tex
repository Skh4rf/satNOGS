\chapter{Perspectives}
\label{chap:persp}
The development of a virtual reality setup brings about exciting possibilities. Hopefully, we will soon see ants express innate behaviours such as searching, navigation, associative learning, etc. without moving in physical space. Coupled with the potential of exploring the neural basis of these behaviours is very promising indeed.
To enable this, I believe work should be targeted in two directions: \\ \\
\textbf{Implement new virtual reality software}. The current software (ViRMEn) has allowed us to probe the use of virtual reality with wood ants, however, it imposes certain limitations in the long term. It is written in Matlab, and customising it has remained a challenge throughout. Furthermore, one is limited to simple virtual environments created within ViRMEn. There are other potential software solutions openly available, e.g. MouseoVeR/FlyOver from Janelia Research Campus \autocite{Cohen2017MouseoVeRLaboratory}. This software is written in the open source programming language Python, and uses environments created in the open source 3D software, Blender. Not only will adopting this solution improve our ability to customise experiments to our needs, it will also improve the reproducibility by being based solely on open source software. \\ \\
\textbf{Develop reward system}. Associative learning experiments entails establishing an association between a cue and a reward (e.g. \cite{Fernandes2017a}), as does traditional navigation experiments with central place foragers (e.g. \cite{Buehlmann2018TheCharacteristics}). To allow such behaviour, the setup needs to allow distribution of reward. This could potentially be accomplished by introducing a syringe with a sucrose solution, however, this approach will first have to be developed and validated before beginning any learning experiments. 