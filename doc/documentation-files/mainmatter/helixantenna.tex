\chapter{Helixantenne}
Die zweite aufgebaute Antenne ist eine monofilare Helixantenne. Dieser Antennentyp ist gerichtet, was bedeutet, dass sie in eine bestimmte Richtung einen höheren Antennengewinn im Vergleich zu anderen Richtungen hat. Sie ist eine der einfachsten Antennenarten um eine zirkulare Polarisation zu erzielen. Kombiniert mit ihrer hohen Bandbreite macht sie das zu einer attraktiven Option zum Nachbau und zur Verwendung in der Satellitenkommunikation.

Für die in dieser Diplomarbeit konstruierte Helixantenne wurden die folgenden Parameter gewählt.

\begin{tabular}{|l|l|}
	\textbf{Parameter} & \textbf{Wert}\\
	Resonanzfrequenz & 433MHz\\
	Windungen & 6\\
	Abstand zwischen Windungen & $0,25\lambda$
\end{tabular}

\section{Design}
Um die Helixantenne zu designen wurden Online-Rechner verwendet REFERENZEN. Folgende Ergebnisse konnten erhalten werden:

GRAFIK

Mithilfe dieser Werte konnte eine erste Näherung für die Antenne konstruiert werden. 

GRAFIK

Diese konnte mithilfe von CENOS-Simulation-Suite simuliert werden. Für die Werte, welche aus dem Rechner erhalten wurden, resultierten folgende Ergebnisse.

S11-Parameter
SWR
Abstrahlverhalten 2D
Abstrahlverhalten 3D

Um die Resonanzfrequenz der Helixantenne zu verschieben wurde der Durchmesser der Spirale verändert. Eine Resonanzfrequenz von 433MHz konnte mit einem Helix-Durchmesser von 270mm erreicht werden. Der Reflektor wurde aus der Theorie mit ca. $\frac{3\lambda}{4}$ festgelegt, was einem Durchmesser von 520mm entspricht.

Es ist wichtig anzumerken, dass der Steigungswinkel der Spirale hierbei ca. 11,5° beträgt. Dies bedeutet, dass die Steigung von dem relativ engen Idealbereich zwischen 12° und 14° abweicht.

\section{Realisierung}
Die Helixantenne besteht theoretisch aus nur zwei Bauteilen: der Spirale und dem Reflektor. Für die Simulation genügte dieses Modell, allerdings werden in der Realität Strukturelemente benötigt um diese zu befestigen.

BILD DER HELIXANTENNE

\begin{itemize}
	\item PVC-Rohr
	\item Abstandhalter (Teflon bzw. PAS60)
	\item Rohrflansch
\end{itemize}

Es lässt sich erkennen, dass 

Aufbau (Teflon Abstandhalter, PVC-Rohr, Spirale, Reflektor) Fusion Modelle, Zeichnungen

Spirale gebogen, Reflektor gelasert, UV-stabilisiertes PVC-Rohr, Befestigungsmethode, Anpassstreifen?, Verschlüsse (3D-Druck), BNC-Buchse, Befestigung des Kabels an der Helix, Befestigung des PVC-Rohrs am Reflektor

\section{Anpassung}
Veränderung der Höhe der Helix, anpassstreifen

\section{Tests}
Tests im Freien, Messungen an Satelliten

\section{Erweiterung der Helixantenne als Array}

\subsection{Gerüst}
effektive apertur, Fusion Modelle, Zeichnungen

\subsection{Anpassung}

\subsection{Test}