\documentclass[12pt]{scrreprt}
\usepackage{babel}
\usepackage{german}
\usepackage{times}
\usepackage{amssymb}
\usepackage{amsmath}
\usepackage{graphicx}
\usepackage{eurosym}

\subject{Diplomarbeit}
\title{Empfangsstation für globales Satellitenbodenstationsnetzwerk SatNOGS}
\author{Ritter Gabriel, Metzler Jakob}
\date{\today}
\publishers{Betreuer: Dipl.-Ing. König Christian}

\begin{document}
	
	\maketitle
	
	\chapter{Abstract}
	Ziel der Diplomarbeit ist es, eine funktionstüchtige Satelliten-Ground-Station aufzubauen, um mit 
	Satelliten im Amateurfunkband, vor allem auch dem CubeSat des STS1, kommunizieren zu können.\\
	
	Im ersten Schritt muss hierzu die Ground-Station selbst aufgebaut werden. Dazu gehören zum 
	Beispiel die Demodulation, Low-Noise-Amplification und ein Software-Defined-Radio. Sobald die 
	Ground-Station funktionstüchtig ist, sollen drei verschiedene Antennen-Typen gebaut und mit der 
	Ground-Station betrieben werden, um den besten Antennen-Typ für den Empfang der CubeSat-Daten zu ermitteln. Die empfangenen Daten sollen weiters über eine grafische Benutzeroberfläche 
	übersichtlich dargestellt werden können.\\
	
	
	Im letzten Schritt werden die verschiedenen Antennen charakterisiert und Werte wie die Richtcharakteristik und der Gain ermittelt.
	
	\chapter{Vorwort}
	Wir wollen allen danken die blabla so dankbar bla
	
	\tableofcontents
	\pagebreak
	
	\chapter{Theoretische Grundlagen}
	In diesem Kapitel werden die theoretischen Grundlagen der Antennen-und Leitertheorie gelegt sowie das SatNOGS Netzwerk näher erläutert.
	
	\section{elektromagnetische Wellentheorie}

	\pagebreak
	
	\chapter{Hauptteil}
	\pagebreak
	
	\chapter{Zusammenfassung}
	\pagebreak
	
	\chapter{Literaturverzeichnis}
	\pagebreak
	
	\chapter{Abkürzungsverzeichnis}
	\pagebreak
	
	\chapter{Begleitprotokoll}
	
	\chapter{Anhang}
	
	

	
\end{document}